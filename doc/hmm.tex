\section{Background}
Hidden Markov models are probabilistic models of sequential data. Such
sequential data could be a stream of phonemes uttered by a speaker or
a stretch of nucleotides. In the case of nucleotides we might be
interested in inferring the position of genes along the sequence. Say we knew that
protein coding regions
were G/C-rich compared to non-coding regions. Then we could model our
stretch of DNA by postulating two hidden states corresponding to
\textit{coding} \& \textit{non-coding}. These two states emit nucleotides with
different probabilities. Moreover, given that the system is in the
coding state, it changes into the non-coding state with some
probability and vice versa. The set of transition probabilities between the
hidden states and emission probabilities attached to each of the hidden states is
called the hidden Markov model.

Given such a model and an unannotated stretch of sequence, we
can try to guess locations of coding and non-coding regions. If, on
the other hand, we do not known the probability parameters
of the model, we can try to infer the model from some ``training''
data using an arbitrary model as our starting point. A thorough
introduction to the algorithms used to solve these problems is
provided in \cite{rab89:mar}. The application of hidden Markov models
in sequence analysis is covered in \cite{dur98:bio}.

\section{Description of Program}
The program allows the user to manipulate the probabilities of a hidden Markov model of a DNA
molecule. The model consists of two hidden states and two observable
states given simulated DNA sequence data. The observable states are A/T and G/C, while the hidden states
are indicated solely by different colors. Given this model, the user can generate a DNA
sequence with hidden states indicated by color coding. Conversely, 
the program can guess the most likely sequence of hidden states given
simulated DNA sequence data. The
inferred hidden states can then be compared to the known states.

In addition, the program can take a simulated DNA sequence and a
random hidden Markov model as input and by an iterative procedure
infer the most likely model. 

\section{Tutorial}
\begin{enumerate}
  \I \ty{Detect Hidden States} Tab
  \begin{enumerate}
    \I The hidden Markov model on the left consists of a starting
    point, which changes into one of the two hidden states
    shown as colored panels with the probabilities indicated on the
    transition arrows. The hidden states each contain a histogram with
    two bars indicating the probabilities with which they emit either
    an A/T or G/C. When emitting A/T, the model emits with equal
    probability an A or a T, and similarly for G/C. Once the model is
    in a certain hidden state it either stays in that state with the
    probability noted next to the self-referential arrow or changes to
    the other state with the probability noted next to the arrow
    pointing to the alternative hidden state. All 10 probabilities
    displayed on the model can be manipulated with the 5 sliders on the
    right. Try out all five sliders and observe the effect.
    \I Reset the model by clicking the reset button.
    \I Generate a DNA sequence by clicking \ty{>}. This sequence appears in the panel on the left hand
    side with
    the hidden states marked by the same colors as they appear in the
    model. 
    \I Estimate the sequence of hidden states from the data by
    clicking \ty{?}. The DNA
    sequence with the estimated annotations appears in the panel on
    the bottom right hand side. Compare the known hidden states with
    those estimated from the data.
  \end{enumerate}
  \I \ty{Estimate HMM} Tab 
  \begin{enumerate}
    \I There are three hidden Markov models displayed here; from left
    to right they indicate:
    \begin{itemize}
      \I The model generated in the \ty{Detect Hidden States} tab.
      \I A random initial model.
      \I The model yet to be estimated. This model is initialized to
      the same values as the random model on its left.
    \end{itemize}
    \I Generate a new DNA sequence by pressing \ty{>}. This
    sequence is generated according to the model displayed on the
    left.
    \I Estimate a new model by pressing \ty{?}. The ensuing
    computations may take some time.
    \I Compare the estimated model to the random and the true
    models. Did you get a good fit? 
    \I Generate a new random model by clicking the die and repeat the model
    estimation. You will find that model estimation depends to some
    extent on the initial model. The reason for this is that the
    estimation algorithm also known as Baum-Welch algorithm can easily
    converge on a local optimum rather than the global optimum, which
    should be very similar to the true model.
  \end{enumerate}
\end{enumerate}





