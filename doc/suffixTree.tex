\section{Background}
Molecular Biologists often want to find sequence patterns in molecular
genetic data. Consider for example the text
$T=\mathtt{ACCGTC}$ and the pattern $P=\mathtt{C}$. A string search
should tell us as efficiently as possible that $P$ occurs in $T$ at
positions 2, 3, and 6. Classical results in computer science have lead
to algorithms that achieve this in time proportional to the length of $T$.

Suffix trees are an ideal data structure for improving on this result in cases
where the text is stable and needs to be searched repeatedly. Suffix
trees are an index structure of the text, which can be built in linear
time \cite{ukk95:alg}. Once built, the text can be searched for
a pattern in time 
proportional to the length of the pattern rather than in time
proportional to the length of the text \cite{gus97:alg}. This is
perhaps not quite as magic as it might initially sound. Consider a
dictionary of the English language such as the wonderful \textit{Oxford English 
  Dictionary}.  Due to its alphabetical structure the
time it takes to find a word in such a large work is relatively independent of its
size. The situation is very different when looking
for, say, \textit{Napoleon} in an unannotated copy of \textit{War and Peace}.

\section{Description of Program}
The program draws a suffix tree for any input string. On this suffix
tree it optionally displays four features:
\begin{enumerate}
  \I The leaf labels, which are the starting positions of the suffices
  constructed by concatenating edge labels from the root of the tree
  to the respective leaf. A suffix of text $T=\mathtt{ACCGTC}$ starts
  somewhere inside $T$ and ends at its end, for example the suffix
  $\mathtt{GTC}$ starts at position 4.
  \I The edge labels, which return one of the text's suffices when read
  from the root to a leaf.
  \I The string depth, which indicate the length of the string
  stretching from the root to the node.
  \I The suffix links, which make it possible for a suffix tree to be
  constructed in linear time \cite{ukk95:alg,gus97:alg}.
\end{enumerate}

\section{Tutorial}
\begin{itemize}
  \I Enter a single letter and construct the corresponding suffix
  tree. Notice that the program always adds the sentinel character
  $\mathtt{\$}$ to
  a string before processing it.
  \I Add a second letter to the input string that is different to the
  first letter and construct the suffix tree.
  \I Continue adding different letters. Notice that the program is case sensitive.
  \I Add a repeated letter and observe the effect on the tree
  topology.
  \I Construct a suffix tree from a string with lots of
  repetitions. Search it for a pattern by matching from the root. If
  the pattern is found in the tree, its start positions in the text are given by the leaves in
  the subtree marked by the end of the match.
  \I Notice that the string read from the root to any internal node
  corresponds to a repeated string in the input sequence. The length
  of this string is indicated by the \textit{string depth}. Look for the longest
  repeat in the input string.
\end{itemize}






