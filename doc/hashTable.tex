\section{Background}
Hash tables are generalized arrays, where instead of using integers as
indices, arbitrary objects are utilized. Such indices
are usually referred to as \textit{keys} and the object accessed though a key
as its \textit{value}. The hashing of query sequences into words of a
predefined length is an important step in some fast database search
tools, including FASTA \cite{pea88:imp}.


\section{Description of Program}
The program takes as input an arbitrary string and computes a table of
words of a certain length and their positions in the text. The word
length can be varied by the user. The program also
displays an indexed version of the input string for easy comparison
with the table entries.

\section{Tutorial}
\begin{itemize}
  \I Hash the default sequence into words of length 1.
  \I Hash the sequence into words of increasing length. What is the
  longest repeat in the input sequence?
  \I Enter a short English sentence (e.g. \textit{O, take the sense,
  sweet, of my innocence!}) and hash it into words of length 3
  by either pressing
  \ty{Enter} or clicking \ty{>}. Do you find a repeated hash key?
\end{itemize}






