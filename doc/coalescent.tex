\section{Background}
Traditional methods for simulating the evolution of a population run
forward in time. In its simplest form such a simulation starts by
creating a population and then transferring it from one generation
to another while implementing the evolutionary scenario under
investigation. Such a scheme has a number of potential problems, including the fact
that a sample drawn at at a certain point in time is not independent
from a sample drawn some time later. Moreover, the representation of
whole populations in the computer is time and memory consuming. 

It is
also unnecessary. In a revolutionary inversion of the perspective on the
evolution of populations, statistical geneticists realized from the
early 1980's onwards  that instead of
simulating populations forward in 
time it is much more efficient to trace the genealogy of samples
backward in time until their most common ancestor is found \cite{kin82:coa,kin82:gen,hud90:gen}. Under this
scheme samples are independent and can be constructed in time
proportional to the sample size rather than the usually much bigger
population size.

When going backward in time, lineages \textit{coalesce} at the point
where they diverged from their last common ancestor. Hence the tree
structure tracing back the genealogy to the most recent common
ancestor of an entire population is called the \textit{coalescent} and
the associated theory is referred to as \textit{coalescent theory}
\cite{hud90:gen,nor01:coa}. 

Notice the difference between the coalescent and a phylogeny. The
latter is reconstructed from empirical data, while the former is a tool for
simulating genetic data.

\section{Description of Program}
The program's toolbar allows the manipulation of the sample size and
of the population parameter theta ($\theta$). This is the scaled
mutation rate
\[
\theta = 2N\mu,
\]
where $N$ is the size of a haploid population and $\mu$ the mutation rate. This
parameter has a simple relationship to the expected number of segregating
sites, $S$,
found in a sample of DNA sequences \cite{wat75:num}:
\begin{equation}\label{eq:s}
S = \theta\sum_{i=1}^{n-1}\frac{1}{i},
\end{equation}
where $n$ is the sample size. Notice that for a sample of two
sequences the number of segregating sites, also known as single
nucleotide polymorphisms, SNPs, is equal to $\theta$. The program
demonstrates the creation of sequence samples with mutations.

\section{Tutorial}
\begin{enumerate}
  \I Click on \texttt{>>} to generate a genealogy. Notice the time
  scale measured in units of $2N$ generations.
  \I Click on \texttt{>>} to throw mutations  onto the
  genealogy. The mutations are displayed as red squares and their average number is given by equation
  (\ref{eq:s}).
  \I Click on \texttt{>>} to generate the haplotypes, or sequences,
  that correspond to the displayed genealogy with mutations. The
  sequences are stylized as labeled black lines with the positions of
  the 
  mutations indicated as before by red squares. Each mutation on the
  genealogy corresponds to a segregating site, i.e. a polymorphic
  column of haplotype positions. Notice that mutations affect all the
  sequences that are located below it in 
  the genealogy. Conversely, mutations on the outer branches of the
  genalogy only appear in a single sequence. These are also
  known as singulets and on average they make up $1/\sum_{i=1}^{n-1}1/i$
  of all the mutations \cite{nor01:coa}
  \I Click the gearwheel to animate the display of random
  genealogies. Notice that you can regulate the speed of animation
  using the \ty{Step Time} slider.
\end{enumerate}





