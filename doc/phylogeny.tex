\section{Background}
The human interest in reconstructing the ancestry of powerful families
is at least as old as the oldest literary documents. For example, 
\textit{Genesis} 5 lists the descendants of Adam down to Noah and 
\textit{Genesis} 10 recounts the descendants of Noah's three sons, who go
on to populate the earth. The preceding description of creation
(\textit{Genesis} 1) is remarkably free of such genealogical thinking.
However, at its most basic level a phylogeny is simply the generalization
of a family tree. 

Phylogenies in the modern sense are as old as Darwin's theory of
evolution (1859), while some of the first algorithms for phylogeny
reconstruction were devised by Edwards \& Cavalli-Sforza
\cite{edw64:rec}. The idea of testing the robustness of a phylogeny
using the bootstrap \cite{efr79:boo} was introduced by Felsenstein in
1985 \cite{fel85:con}. The large field of phylogenetics is surveyed in
\cite{swo96:phy,pag98:mol,fel04:inf}. 


\section{Description of Program}
The program takes as input a set of aligned sequences in FASTA
format and seven primate mitochondrial DNA sequences are provided as
default input. The program computes the pairwise distances between
these sequences. Users can choose between three different distance
measures:
\begin{enumerate}
  \I The number of mismatches
  \I The number of mismatches normalized by division by the length of the alignment 
  \I The Jukes-Cantor distance \cite{juk69:evo}
\end{enumerate}

A simple phylogenetic tree is reconstructed from these distances using the
 UPGMA (unweighted pair group method with arithmetic means)
method \cite{sok63:pri}. In addition, the program demonstrates the
 application of the 
bootstrap in phylogeny reconstruction.

\section{Tutorial}
\begin{enumerate}
  \I \ty{Distance Matrix} Tab
  \begin{enumerate}
    \I Compute the pairwise number of mismatches between the displayed
    sequences. 
    \I Count the number of mismatches between the human \& chimp
    sequences and compare your result with the corresponding entry in
    the distance matrix.
    \I Compute the normalized number of mismatches and note down the
    distance between human and gibbon.
    \I Compute the Jukes-Cantor distance. Compare the Jukes-Cantor
    distance between human and gibbon with the normalized number of
    mismatches for this pair of taxa.
  \end{enumerate}
  \I \ty{Phylogeny} Tab
  \begin{enumerate}
    \I Compute the phylogeny from the distance data.
    \I Display the brach lengths on the tree.
    \I Compare the branch lengths with the distances in the distance
    matrix. 
  \end{enumerate}
  \I \ty{Bootstrap} Tab
  \begin{enumerate}
    \I Generate a bootstrap sample from the original data. This is
    done by sampling with replacement columns from the original
    alignment.
    \I Compute the distance matrix for the bootstrap data.
    \I Generate the phylogenetic tree for the new distance matrix. Is
    the tree topology the same as for the original data set?
    \I Repeat the process of bootstrapping and observe the range of
    tree topologies generated.
    \I Click the cogwheel to animate the generation of bootstrapped
    trees. Notice that you can vary the animation speedy using the
    \ty{Step Time} slider. 
  \end{enumerate}
\end{enumerate}





