\section{Background}
A dot plot is a simple means of comparing two strings. Consider for
example the following string in jest ascribed to Napoleon I (1769--1821):
\begin{center}
  \ty{ABLEWASIEREISAWELBA}
\end{center}
If we compare this string to itself and make a dot every time we find
a matching character, we obtain the graphic shown in Figure
\ref{fig:dot}.

\begin{figure}
\begin{center}\resizebox{\textwidth}{!}{\begin{tabular}{|c||c|c|c|c|c|c|c|c|c|c|c|c|c|c|c|c|c|c|c|}\hline
&\color{red}A &\color{red}B &\color{red}L &\color{red}E &\color{red}W &\color{red}A &\color{red}S &\color{red}I &\color{red}E &\color{red}R &\color{red}E &\color{red}I &\color{red}S &\color{red}A &\color{red}W &\color{red}E &\color{red}L &\color{red}B &\color{red}A\\ \hline\hline
\color{red}A &\color{blue} $\bullet$ & & & & &\color{blue} $\bullet$ & & & & & & & &\color{blue} $\bullet$ & & & & &\color{blue}$\bullet$ \\\hline
\color{red}B & &\color{blue} $\bullet$ & & & & & & & & & & & & & & & &\color{blue} $\bullet$ & \\\hline
\color{red}L & & &\color{blue} $\bullet$ & & & & & & & & & & & & & &\color{blue} $\bullet$ & & \\\hline
\color{red}E & & & &\color{blue} $\bullet$ & & & & &\color{blue} $\bullet$ & &\color{blue} $\bullet$ & & & & &\color{blue} $\bullet$ & & & \\\hline
\color{red}W & & & & &\color{blue} $\bullet$ & & & & & & & & & &\color{blue} $\bullet$ & & & & \\\hline
\color{red}A &\color{blue} $\bullet$ & & & & &\color{blue} $\bullet$ & & & & & & & &\color{blue} $\bullet$ & & & & &\color{blue}$\bullet$ \\\hline
\color{red}S & & & & & & &\color{blue} $\bullet$ & & & & & &\color{blue} $\bullet$ & & & & & & \\\hline
\color{red}I & & & & & & & &\color{blue} $\bullet$ & & & &\color{blue} $\bullet$ & & & & & & & \\\hline
\color{red}E & & & &\color{blue} $\bullet$ & & & & &\color{blue} $\bullet$ & &\color{blue} $\bullet$ & & & & &\color{blue} $\bullet$ & & & \\\hline
\color{red}R & & & & & & & & & &\color{blue} $\bullet$ & & & & & & & & & \\\hline
\color{red}E & & & &\color{blue} $\bullet$ & & & & &\color{blue} $\bullet$ & &\color{blue} $\bullet$ & & & & &\color{blue} $\bullet$ & & & \\\hline
\color{red}I & & & & & & & &\color{blue} $\bullet$ & & & &\color{blue} $\bullet$ & & & & & & & \\\hline
\color{red}S & & & & & & &\color{blue} $\bullet$ & & & & & &\color{blue} $\bullet$ & & & & & & \\\hline
\color{red}A &\color{blue} $\bullet$ & & & & &\color{blue} $\bullet$ & & & & & & & &\color{blue} $\bullet$ & & & & &\color{blue}$\bullet$ \\\hline
\color{red}W & & & & &\color{blue} $\bullet$ & & & & & & & & & &\color{blue} $\bullet$ & & & & \\\hline
\color{red}E & & & &\color{blue} $\bullet$ & & & & &\color{blue} $\bullet$ & &\color{blue} $\bullet$ & & & & &\color{blue} $\bullet$ & & & \\\hline
\color{red}L & & &\color{blue} $\bullet$ & & & & & & & & & & & & & &\color{blue} $\bullet$ & & \\\hline
\color{red}B & &\color{blue} $\bullet$ & & & & & & & & & & & & & & & &\color{blue} $\bullet$ & \\\hline
\color{red}A &\color{blue} $\bullet$ & & & & &\color{blue} $\bullet$ & & & & & & & &\color{blue} $\bullet$ & & & & &\color{blue}$\bullet$ \\\hline
\end{tabular}}\end{center}
  \caption{Dot plot of the string \ty{ABLEWASIEREISAWELBA} compared to
  itself.}
  \label{fig:dot}
\end{figure}

Notice that the main diagonal running from the top left corner to the
bottom right corner is filled with dots because the string is compared
to itself. In addition, the counter diagonal is also filled with dots,
because the string is a palindrome.

A simple generalization of the kind of dot plot shown in Figure
\ref{fig:dot} is achieved by plotting matches of some length $\ge 1$
and this approach has been used to compare sequences on a genomic
scale \cite{son96:dna}.

\section{Description of Program}
The program takes two sequences as input and displays matches of a
certain length. The user can enter new input sequences and vary the
length of the matches displayed. In addition, the user can randomize
the input sequences to observe the length of matches between random
sequences.

\section{Tutorial}
\begin{enumerate}
  \I Click \ty{>} to draw a dot plot of the two alcohol dehydrogenase sequences
  supplied by the program using a \ty{Match Length} of 11.  
  \I Observe the matches,
  indicated as black lines, along the diagonal.  
  \I Notice the gap in
  the matches, which is caused by the insertion of the copia
  transposon in Sequence 1 at that position (c.f. header line of Sequence 1).
  \I Randomize the two sequences by clicking the die and redraw the dot plot. 
  \I Vary the \ty{Match Length} to find the longest match between the
  two randomized sequences.
\end{enumerate}





