\section{Background}
Pairs of residues in DNA alignments are traditionally scored by
distinguishing only between matches and mismatches. This works
reasonably well for DNA sequences but is not applicable to
proteins. For a start, the number of mutational steps needed to
convert a codon for one amino acid into that for another varies
between one and three. In addition, amino acids differ markedly in their
physico-chemical properties. As a result, amino acid substitutions
have highly variable effects on the structure of the protein
affected. Since most changes in a protein's structure are deleterious,
selection will filter out 
substitutions with drastic structural effects. Substitution matrices
summarize the $21\times20/2 = 210$
log-likelihoods of observing a homologous pair of amino acids. 
Finally, substitution probabilities change over time,
making it necessary to construct series of substitution
matrices to cover a range of evolutionary time scales. The first such
series that is still widely used today is known as the PAM
series and was developed in the 1970's by Margaret Dayhoff and coworkers
\cite{day78:mod}. PAM stands for \textbf{p}ercent \textbf{a}ccepted
\textbf{m}utations and hence a given PAM$n$
matrix is applicable to pairs of sequences with an expected evolutionary
distance of $n$ mutations per 100 residues.

A more recent series of protein substitution matrices is based on 
blocks of aligned protein sequences and hence known as \textbf{BLO}cks
\textbf{SU}bstitution \textbf{M}atrices or BLOSUM
\cite{hen92:ami}. In contrast PAM matrices, which are based on
extrapolating from a single set of observed substitution
probabilities, BLOSUM matrices are each constructed directly from
aligned sequence data. In order to construct a BLOSUMx matrix,
sequences that are $\ge x\%$ similar are clustered and given equal
weight in the computations. Notice that increasing PAM numbers imply
increasing evolutionary distances while the converse is true for
BLOSUM matrices. Table \ref{tab:pam} shows corresponding BLOSUM and
PAM matrices. 
\begin{table}
\begin{center}
\begin{tabular}{cc}
\hline\hline
PAM & BLOSUM\\
\hline
PAM 120 & BLOSUM 80\\
PAM 160 & BLOSUM 62\\
PAM 250 & BLOSUM 45\\
\hline\hline
\end{tabular}
\end{center}
\caption{Corresponding substitution matrices from the PAM and BLOSUM series.}\label{tab:pam}
\end{table}

\section{Description of Program}
The program displays the PAM and BLOSUM matrices. On the \textbf{PAM
  tab} PAM matrices are
provided for evolutionary distances ranging
from 2 PAM to 500 PAM. The user can select the PAM
number through a slider. The entries of PAM matrices are expressed in bits and by
convention different bit fractions have been used to scale the
majority of matrix entries to single digit integers. The user can therefore choose
from a range of such scale factors. In addition, the program displays
the expected percentage of mismatched residues between proteins
separated by an evolutionary distance of a given PAM.
The implemented calculations are based on the original data
\cite{day78:mod} and hence the matrices may slightly diverge in
numerical detail from the versions currently available, e.g. as part
of the \ty{BLAST} distribution \cite{alt97:gap}.

On the \textbf{BLOSUM tab} three commonly used BLOSUM matrices can be
displayed: BLOSUM 45, 62, and 80.


\section{Tutorial}
\begin{enumerate}
  \I Look at the PAM matrix displayed when the program is first
  launched and notice that the values along
  the main diagonal, which represent match scores, are all
  positive. This contrasts with the off-diagonal entries, which
  represent mismatch scores. These are on
  average negative. Identify the two amino acids with the highest entries on
  the main diagonal, i.e. the two most conserved amino
  acids. Can you make biological sense of their high degree of conservation?
  \I Slide the PAM number to small values. Notice that the
  \%-difference is now equal to the PAM number. This is because, say,
  2 Mutations per 100 amino acids will on average have caused a 2\%
  difference in the corresponding pairwise comparison.
  \I Slide the PAM number to very high values. Notice three things:
  \begin{enumerate}
    \I The PAM number now vastly exceeds the \%-difference. This is
    because mutations can hit a given position more than once and this 
    occurs with increasing frequency the more mutations are thrown
    randomly at a segment of 100 amino acids. At a more basic level,
    percentages vary between 0 and 100, while the number of mutations
    per 100 residues (PAM$n$) 
    can range between 0 and infinity.
    \I The entries in the matrix tend towards zero. The reason for
    this is that the probability of finding the corresponding pair of
    residues due
    to homology becomes over long evolutionary time equal to the
    probability of finding the pair by chance. The ratio of two equal
    numbers is 1 and $\log{1} = 0$.
    \I A few amino acids retain high entries on the main diagonal even
    for high PAM entries. Identify the two amino acids with the
    highest entries.
  \end{enumerate}
  \I Compare the entries between PAM160 and BLOSUM62 for amino acid
  pairs histidine/alanine, valine/valine, and
  tryptophan/tryptophan. Do you notice any differences? 
\end{enumerate}





