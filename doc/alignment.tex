\section{Background}
Similar sequences tend to encode similar functions. The search for
sequence similarities is therefore the bread and butter of
bioinformatics. This program is concerned with alignment by 
dynamic programming, which was pioneered
for global alignment by Needleman and Wunsch \cite{nee70:gen}. The local
alignment algorithm is due to Smith and Waterman
\cite{smi81:ide}.

\section{Description of Program}
This program visualizes the dynamic programming matrix used for
computing optimal pairwise alignments between two sequences over an
arbitrary alphabet. It also visualizes the path through the matrix
that corresponds to an optimal alignment. In addition, it displays the
alignment, its score, and calculates
the number of co-optimal global alignments. Three alignment modes are
implemented:
\begin{enumerate}
  \I Global: Sequences are assumed to be homologous over their entire
  length.
  \I Local: Sequences are assumed to possess only local homology.
  \I Overlap: Sequences are assumed to have overlapping ends. This
  approach is used for the assembly of genome sequences from
  shotgun fragments \cite{gus97:alg}.
\end{enumerate}

The user can enter any sequence and manipulate the following
parameters:
\begin{enumerate}
  \I Match score; case sensitive
  \I Mismatch score; case sensitive
  \I Gap extension
\end{enumerate}

Notice that most alignment algorithms implement an affine gap scoring
scheme, i.e. the total gap cost, $g_{\rm t}$ is
\[
g_{\rm t} = g_{\rm o} + g_{\rm e}\times l,
\]
where $g_{\rm o}$ is the gap opening cost, $g_{\rm e}$ the gap
extension cost, and $l$ the length of the gap. In this program the gap
opening cost is ignored, i.e. set to zero.

\section{Tutorial}
\begin{enumerate}
  \I Choose the global alignment algorithm.
  \I Click on \texttt{>>} to set up the dynamic programming matrix
  corresponding to the sequences entered.
  \I Click again on \ty{>>} to initialize the dynamic programming
  matrix.
  \I Click on \ty{>>} to fill in the dynamic programming
  matrix. Verify that the entry in cell $(\mathtt{A},\mathtt{A})$ has the expected
  value. 
  \I Click one final time on \ty{>>} to carry out the trace back. The
  corresponding alignment and its score is shown below. In addition,
  the number of co-optimal alignments is displayed. These are
  alternative paths through the matrix carrying the same optimal
  score. There is no algorithmic criterion for distinguishing between
  these alternatives.
  \I Repeat the above steps using the local alignment algorithm.
  \I Repeat the above steps using the overlap alignment algorithm.
\end{enumerate}





