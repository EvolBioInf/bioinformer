\section{Background}
At its most basic, string matching consists of looking for a short
pattern in a large text. This is a classical topic of computer science
\cite{gus97:alg}. However, string matching
methods are ubiquitous in bioinformatics, since a core concern of the
field is the interpretation of genetic sequence texts
\cite{gus97:alg}. 

\section{Description of Program}
The program implements three methods of string matching:
\begin{enumerate}
  \I \textbf{Naive string matching}, which essentially consists of two nested
  \ty{do} loops, the outer running over the text, the inner over the
  pattern. Accordingly, the worst case run time of this algorithm is $O(n\times
  m)$, where $m$ and $n$ are the lengths of the text and the pattern,
  respectively. 
  \I The \textbf{Z-Algorithm} is thought to be the simples string
  matching algorithm that runs in time linear in the length of the
  text \cite{gus97:alg}.
  \I A \textbf{suffix tree} is a data structure for indexing a
  text. Once this data structure is built, it has the astonishing
  property that a text can be searched in time
  proportional to the length of the pattern. Building the suffix tree
  itself only 
  takes time proportional to the length 
  of the text, but the factor of
  proportionality is rather large \cite{ukk95:alg,gus97:alg}.
\end{enumerate}
To demonstrate the time requirements of the three algorithms the
program displays the time taken for searching and --- with the exception
of the naive algorithm, which lacks preprocessing --- the time taken for
preprocessing. The text window can be freely edited. In addition,
three chunks of 
interesting prose are provided:
\begin{enumerate}
  \I The sequence of the first genome to be sequenced, that of the
  bacteriophage $\Phi$-X174, which encodes 10
  proteins in 5386 nucleotides \cite{air78:nuc}.
  \I The alcohol dehydrogenase (\textit{Adh}) region of
  \textit{Drosophila melanogaster}. In 1983 this was the first gene to
  be subjected to comparative sequencing \cite{kre83:nuc}. 
  \I The text of the paper by Watson \& Crick announcing the
  double helical structure of DNA \cite{wat53:mol}.
\end{enumerate}

\section{Tutorial}
\begin{itemize}
  \I Choose the \ty{Naive} algorithm in conjunction with the genome of
  $\Phi$-X174 and 
  run the pattern search for all three algorithms provided. Observe in
  particular the dramatic variations in preprocessing time.
\end{itemize}






